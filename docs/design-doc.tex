\documentclass[10pt, a4paper, conference]{IEEEtran}
%\IEEEoverridecommandlockouts
% The preceding line is only needed to identify funding in the first footnote. If that is unneeded, please comment it out.
\usepackage{cite}
\usepackage{amsmath,amssymb,amsfonts}
\usepackage{algorithmic}
\usepackage{graphicx}
\usepackage{textcomp}
\usepackage{xcolor}
\def\BibTeX{{\rm B\kern-.05em{\sc i\kern-.025em b}\kern-.08em
    T\kern-.1667em\lower.7ex\hbox{E}\kern-.125emX}}
\begin{document}

\title{Implementing Kalah with a Changeable Object-Oriented Design\\
}

\author{\IEEEauthorblockN{Matthew Eden}
\IEEEauthorblockA{\textit{Department of Electrical, Computer and Software
Engineering} \\
\textit{University of Auckland}\\
Auckland, New Zealand \\
mede607@aucklanduni.ac.nz}
}

\maketitle

\begin{abstract}
There were several decisions made in designing Kalah (a.k.a Mancala) in Java
  7 such that it supported changeability. The main aspects of the design
    considered 
\end{abstract}

\begin{IEEEkeywords}
kalah, mancala, object-oriented design, changeability, java
\end{IEEEkeywords}

\section{Introduction}
This document outlines the design decisions made when designing an
implementation of the game Kalah (known elsewhere as Mancala) in Java 7.
There were three aspects of design considered during development, and these
  aspects were considered with great concern as to how they affect the
  changeability of the object-oriented design.
The first aspect of the design concerned the number of 'houses' available to
  a player. The prescribed set of rules dictated \textit{six} of these 'houses'
  for each player, however it was decided that this number should be
  changeable to accomodate future rulesets.
The second aspect of the design concerned the number of 'seeds' that each house
  starts the game with. The prescribed set of rules dicated that this starting
  value should be \textit{4} seeds per house, however the design should allow
  for ruleset changes.
The third aspect of the design concerned the separation of input/output and
game logic. This is to ensure future alterations to the specification of the
output that was originally provided can be executed without needing to consider
implications on the functionality of the game itself.

\section{Design Decisions}

\subsection{Number of Houses}
The design I have implemented allows any number of houses (a minimum of
1 house) to be used. This supports changeability

\subsection{Number of Seeds}
My design allows a developer to easily alter the number of seeds in each house
at the start of the game. In addition, there is also support for altering the amount of seeds that are dropped into each pit when a player makes a move.

\subsection{Separation of IO and Game Logic}
My design makes use of a class \textit{GameIO}, which 
is concerned with handling player input and console output.
This supports changeability as it means future alterations to IO are
constrained to a single class. 

%\begin{thebibliography}{00}
%\bibitem{b1} G. Eason, B. Noble, and I. N. Sneddon, ``On certain integrals of Lipschitz-Hankel type involving products of Bessel functions,'' Phil. Trans. Roy. Soc. London, vol. A247, pp. 529--551, April 1955.
%\end{thebibliography}

\end{document}
