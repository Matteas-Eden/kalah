\documentclass[10pt, a4paper, conference]{IEEEtran}
%\IEEEoverridecommandlockouts
% The preceding line is only needed to identify funding in the first footnote. If that is unneeded, please comment it out.
\usepackage{cite}
\usepackage{amsmath,amssymb,amsfonts}
\usepackage{graphicx}
\usepackage{textcomp}
\usepackage{xcolor}
\def\BibTeX{{\rm B\kern-.05em{\sc i\kern-.025em b}\kern-.08em
    T\kern-.1667em\lower.7ex\hbox{E}\kern-.125emX}}
\begin{document}

\title{Implementing Kalah with a Changeable Object-Oriented Design\\
}

\author{\IEEEauthorblockN{Matthew Eden}
\IEEEauthorblockA{\textit{Department of Electrical, Computer and Software
Engineering} \\
\textit{University of Auckland}\\
Auckland, New Zealand \\
mede607@aucklanduni.ac.nz}
}

\maketitle

\begin{abstract}
There were several decisions made in designing Kalah (a.k.a Mancala) in Java
  7 such that it supported changeability. Three of those decisions are
  discussed here: the creation of a configuration class, the abstraction of
  game elements and the separation of IO from game logic.
\end{abstract}

\begin{IEEEkeywords}
kalah, mancala, object-oriented design, changeability, java
\end{IEEEkeywords}

\section{Design Decisions}

Associate Professor Ewan Tempero defined changeability as 
\textit{How much it costs to make the necessary changes to existing code, once those
changes have been identified} \cite{changeability}. The design decisions
outlined here were made with this specific definition in mind.

\subsection{Use of a Config class}
The design I have implemented groups commonly used values inside a globally
accessible configuration class \texttt{GameConfig}, such as
\texttt{NUM\_HOUSES} and \texttt{STARTING\_SEEDS}. These values can be modified
within the class to then alter the behaviour of the game, such as increasing
the number of houses for each player.
This supports changeability because the cost of changing these values is very
low, given they are merely constants and although they are referenced in many
places, they are only defined once.

\subsection{Abstraction of Game Elements}
My design abstracts different game elements into dedicated classes, such as
\texttt{Player}, \texttt{Board} and \texttt{Pit} (a combination of house and
store). This decision supports changeability as an individual looking to change
these aspects of the game can do so by only changing the implementations of
these classes. By having classes with high cohesion, the cost of a change can
be reduced.

\subsection{Separation of IO and Game Logic}
My design makes use of a class \texttt{GameIO}, which 
is concerned with handling player input and console output.
This decision supports changeability as it means future alterations to IO are
constrained to a single class. It also means that an individual does not need
to be concerned with breaking the functionality of the game by altering IO.

\begin{thebibliography}{00}
\bibitem{changeability} E. Tempero ``SOFTENG 701 - Lecture 01b
  - Changeability'' Canvas, slide 12, May 2020.
\end{thebibliography}

\end{document}
