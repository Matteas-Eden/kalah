\documentclass[10pt, a4paper, conference]{IEEEtran}
%\IEEEoverridecommandlockouts
% The preceding line is only needed to identify funding in the first footnote. If that is unneeded, please comment it out.
\usepackage{cite}
\usepackage{amsmath,amssymb,amsfonts}
\usepackage{graphicx}
\usepackage{textcomp}
\usepackage{xcolor}
\def\BibTeX{{\rm B\kern-.05em{\sc i\kern-.025em b}\kern-.08em
    T\kern-.1667em\lower.7ex\hbox{E}\kern-.125emX}}
\begin{document}

\title{Implementing Kalah with a Changeable Object-Oriented Design\\
}

\author{\IEEEauthorblockN{Matthew Eden}
\IEEEauthorblockA{\textit{Department of Electrical, Computer and Software
Engineering} \\
\textit{University of Auckland}\\
Auckland, New Zealand \\
mede607@aucklanduni.ac.nz}
}

\maketitle

\begin{abstract}
There were several decisions made in designing Kalah (a.k.a Mancala) in Java
  7 such that it supported changeability. Three of those decisions are
  discussed here: the creation of a configuration class, the abstraction of
  game elements and the separation of IO from game logic.
\end{abstract}

\begin{IEEEkeywords}
kalah, mancala, object-oriented design, changeability, java
\end{IEEEkeywords}

%\section{Introduction}
%This document outlines the design decisions made when designing an
%implementation of the game Kalah (known elsewhere as Mancala) in Java 7.
%There were three aspects of design considered during development, and these
%  aspects were considered with great concern as to how they affect the
%  changeability of the object-oriented design.
%The first aspect of the design concerned the number of 'houses' available to
%  a player. The prescribed set of rules dictated \textit{six} of these 'houses'
%  for each player, however it was decided that this number should be
%  changeable to accomodate future rulesets.
%The second aspect of the design concerned the number of 'seeds' that each house
%  starts the game with. The prescribed set of rules dicated that this starting
%  value should be \textit{4} seeds per house, however the design should allow
%  for ruleset changes.
%The third aspect of the design concerned the separation of input/output and
%game logic. This is to ensure future alterations to the specification of the
%output that was originally provided can be executed without needing to consider
%implications on the functionality of the game itself.

\section{Design Decisions}

Associate Professor Ewan Tempero defined changeability as 
\textit{How much it costs to make the necessary changes to existing code, once those
changes have been identified} \cite{changeability}. The design decisions
outlined here were made with this specific definition in mind.

\subsection{Use of a Config class}
The design I have implemented groups commonly used values, such as
\texttt{NUM\_HOUSES} and \texttt{STARTING\_SEEDS}, inside a globally
accessible configuration class \texttt{GameConfig}. These values can be modified
within the class to then alter the behaviour of the game, such as increasing
the number of houses for each player.
This decision supports changeability because the cost of changing these values is very
low (given they are merely constants) and although they are referenced in many
places, they are only defined once. As such, if an individual wishes to change
the value of some parameter, like the number of houses per player, they need
only change this value at its definition in the configuration class instead of
at every point in the code where that value is used.

\subsection{Abstraction of Game Elements}
My design abstracts different game elements into dedicated classes, such as
\texttt{Player}, \texttt{Board} and \texttt{Pit} (a combination of house and
store). Each of these classes has been designed in such a way to reduce
coupling and increase cohesion. For example, there is a single method in 
\texttt{Board}, \texttt{makeMove}, which is concerned with the logic of 
a player's move.
This decision supports changeability as an individual looking to change
these aspects of the game can do so by only changing the implementations of
these classes or methods contained within these classes. As such, the cost
of a change to these aspects of the game is minimised.

\subsection{Separation of IO and Game Logic}
My design makes use of a class \texttt{GameIO}, which 
is concerned with handling player input and console output. It is a dedicated
class; that is to say outside of it there are no other invocations of IO
methods. 
This decision supports changeability as it means future alterations to IO are
constrained to a single class. It also means that an individual does not need
to be concerned with breaking the functionality of the game by altering IO, as
they have been separated in this way, thus lowering the cost of the change.

\begin{thebibliography}{00}
\bibitem{changeability} E. Tempero ``SOFTENG 701 - Lecture 01b
  - Changeability'', Slide 12 of 17, SOFTENG 701 2020, canvas.auckland.ac.nz,
    retrieved on May 13\textsuperscript{th} 2020.
\end{thebibliography}

\end{document}
