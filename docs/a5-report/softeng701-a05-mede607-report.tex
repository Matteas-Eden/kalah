\documentclass[10pt, a4paper, conference]{IEEEtran}
%\IEEEoverridecommandlockouts
% The preceding line is only needed to identify funding in the first footnote. If that is unneeded, please comment it out.
\usepackage{cite}
\usepackage{amsmath,amssymb,amsfonts}
\usepackage{graphicx}
\usepackage{textcomp}
\usepackage{xcolor}
\def\BibTeX{{\rm B\kern-.05em{\sc i\kern-.025em b}\kern-.08em
    T\kern-.1667em\lower.7ex\hbox{E}\kern-.125emX}}
\begin{document}

\title{Implementing the \textit{Best move or first robot} change case}

\author{\IEEEauthorblockN{Matthew Eden}
\IEEEauthorblockA{\textit{Department of Electrical, Computer and Software
Engineering} \\
\textit{University of Auckland}\\
Auckland, New Zealand \\
mede607@aucklanduni.ac.nz}
}

\maketitle

\begin{abstract}
  To evaluate the changeability of an implementation of Kalah (a.k.a Mancala), a change case was
  proposed. This involved replacing the second player in the game with a simple
  AI dubbed \textit{Best move or first robot}. Instead of taking player inputs,
  this AI would determine moves based on whether it was possible to achieve an
  extra move or a capture. If neither were possible, it simply chose the first
  available house (i.e. first non-empty house). A change plan was created in anticipation
  of implementing the changes in code. During the execution of this plan, there
  were several issues encountered as a result of the complexity of the change
  case and the somewhat lacking changeability of the design requiring much new
  code to be written. The time taken to implement the change case was roughly 50 minutes.
  The overall impact of the change case was mild, as the base ruleset hasn't
  changed but the game is noticeably different. 
\end{abstract}

\begin{IEEEkeywords}
kalah, mancala, changeability, java
\end{IEEEkeywords}

\section{Planning the Change Case}
Before any code was actually written, a plan was devised describing the minimal
amount of changes that would be required to achieve the desired change case.
This plan is as follows.
\begin{enumerate}
  \item Find the \texttt{GameConfig} class.
  \item Change the value of \texttt{VERTICAL\_OUTPUT} from \texttt{true} to
    \texttt{false}.
  \item Find the \texttt{GameIO} class
  \item Create a new method: \texttt{public static int computeMove(Board board,
    int playerNum, IO io)}.
  \item Within this method, create a \texttt{List<Pit>} to contain the player's
    houses returned from
    \texttt{board.getPlayers().get(playerNum).getHouses()}.
  \item Create three separate loops to iterate through each object in that
    list.
  \item In the first loop, implement logic to determine if a move exists that
    leads to an extra move. Achieve this by comparing the number of seeds that
    would be sown and the distance to the player's store. If equal, that move
    leads to an extra move. Otherwise it does not.
  \item In the second loop, implement logic to determine if a move exists that
    leads to a capture. Achieve this by calculating where the last seed sown
    will land and using this information to determine whether a capture occurs
    at that location.
  \item In the third loop, implement logic to determine the legal move that
    exists with the lowest house number i.e. index.
  \item Find the \texttt{Kalah} class.
  \item Find the \texttt{Play} method.
  \item Within the \texttt{while} loop, modify the assignment to the variable
    \texttt{selection} to read: \texttt{selection = (playerNum == 1)
    ? GameIO.computeMove(board, playerNum, io) : GameIO.getMove(board,
    playerNum, io);}.
\end{enumerate}

\section{Implementing the Change Case}
The total amount of time taken to implement the change case was timed as being 
48 minutes and 50 seconds. During the execution of the aforementioned plan,
there were various issues encountered that arose due to the complexity of
the change case. In particular, implementing the base logic for the BMF bot
proved troublesome and the change plan did not convey in enough detail how
this should be done; thereby requiring me as a developer to figure it out on
the spot. Although there was a lot of new code to be written, a large majority
of that code was constrained to a single method which made it easier to
manage. A breakdown of each step in the change plan follows.
\begin{enumerate}
  \item No issues in finding the \texttt{GameConfig} class.
  \item No issues in changing the value of \texttt{VERTICAL\_OUTPUT}.
  \item No issues in finding the \texttt{GameIO} class.
  \item No issues in creating the \texttt{computeMove} method.
  \item No issues in creating a \texttt{List<Pit>} or invoking the required
    methods to obtain a value for it.
  \item No issues in creating the loops.
  \item In the first loop, some difficulties arose with implementing the
    comparison condition. Particularly with regard to moves that sow enough
    seeds to circle around the board, passing both player's houses.
  \item In the second loop, it was difficult to calculate the position at which
    the last seed of a potential move would be sown. Particularly with regard
    to how the opposing player's store is skipped when sowing seeds that circle
    the board.
  \item No issues in implementing the logic for the third loop.
  \item No issues in finding the \texttt{Kalah} class.
  \item No issues in finding the \texttt{Play} method.
  \item No issues in modifying the assignment of \texttt{selection}.
\end{enumerate}

\section{Impact of the Change Case}
On a scale from \textit{0.0} to \textit{1.0}, the impact of this change case is
\textit{0.3}. Although the rules of \textit{Kalah} are preserved and the
overall functionality of the game remains largely unchanged, the replacement 
of the second player with a simple AI changes the game from being multi-player 
to being single-player.

\section{Changeability Assessment}
Implementing this change case required a large amount of new code to be
written, and this exposed some gaps in the changeability of the design. This
change case was implemented on top of a previous change case
\cite{a4-change-case}, and in that instance the design showed a good level of 
changeability. However, due to a lack of factorisation of functionality related
to determining different move outcomes (i.e. gaining an extra move or executing
a capture), this aspect of the design did not have good changeability and that
made this change case more difficult to implement. If a future developer wishes 
to change the conditions that lead to an extra move or a capture they will need 
to significantly alter the design and doing so will not be a straightforward process, 
as I have discovered in implementing this change case. So, contrary to claims made 
previously \cite{a3-design}, this design of Kalah only \textit{partially} lends itself 
to changeability and could be improved to be more changeable.

\begin{thebibliography}{00}
  \bibitem{a3-design} M. Eden ``SOFTENG 701 - Assignment 3 - Submission``,
    \textit{Implementing Kalah with a Changeable Object-Oriented Design}
  \bibitem{a4-change-case} M. Eden ``SOFTENG 701 - Assignment 4 - Submission``,
    \textit{Implementing the Change Board Orientation change case}
\end{thebibliography}
\end{document}
