\documentclass[10pt, a4paper, conference]{IEEEtran}
%\IEEEoverridecommandlockouts
% The preceding line is only needed to identify funding in the first footnote. If that is unneeded, please comment it out.
\usepackage{cite}
\usepackage{amsmath,amssymb,amsfonts}
\usepackage{graphicx}
\usepackage{textcomp}
\usepackage{xcolor}
\def\BibTeX{{\rm B\kern-.05em{\sc i\kern-.025em b}\kern-.08em
    T\kern-.1667em\lower.7ex\hbox{E}\kern-.125emX}}
\begin{document}

\title{Reflecting on Changeability through Kalah}

\author{\IEEEauthorblockN{Matthew Eden}
\IEEEauthorblockA{\textit{Department of Electrical, Computer and Software
Engineering} \\
\textit{University of Auckland}\\
Auckland, New Zealand \\
mede607@aucklanduni.ac.nz}
}

\maketitle

\begin{abstract}
  As part of an investigation into changeability, I was tasked with creating an
  implementation of Kalah that followed an object-oriented design and aimed to
  have high changeability. This design was then subjected to 2 change cases,
  one concerned with the game output and another concerned with replacing one
  player with a bot. Through these change cases I gained some understanding of
  changeability and what it means in practice.
\end{abstract}

\begin{IEEEkeywords}
kalah, mancala, changeability, reusability, java, solid, object-oriented design
\end{IEEEkeywords}

\section{Introduction}
To investigate changeability and better understand the abstract ideas presented
in the course \textit{SOFTENG 701 - Advanced Software Engineering Development
Methods}, I was tasked with creating an implemention of the traditional game
\textit{Kalah} in Java 7. The aims of this design was to have high
changeability, where changeability is defined as ``How much it costs to make
the necessary changes to existing code (not new code) once those changes have
been identified'' \cite{lec01b-changeability}. The changeability of this design
was then tested through two change cases, the first of which was concerned with
the display of the game and the second with the actions of the player.

\section{Designing for Changeability}
Looking back on my original design, there are several aspects of it that do not
hold up to scrutiny when considering changeability. However, this is to be
expected as at that point in time I did not have a good formal understanding of
changeability, as it has been presented in this course. Previously, if I had
been asked about how one might design code such that it is changeable, the
advice I would have given would be in regards to ensuring atomicity in methods
and classes. I would give this advice with the mindset that a changeable design
is one that is `pluggable'; new functionality can be added in without much
change to the current design and elements of the design can be taken out and
put into other designs. Although this advice does sound good at face value, and
has done well for me thus far, it lacks formality. In contrast to the view of
changeability presented in this course, I wouldn't have made mention of
a \textit{changeability index}, or considered that a design is still
changeable regardless of how much new code is written for a change case. I also
would not have considered much of the advice presented in this course, such as
design patterns or the SOLID principles.

Having now gotten to a point where I am aware of this advice, I can start to
see some flaws in my original design. Particularly in regards to SOLID
principles, such as the \textit{Single Responsibility Principle}
\cite{lec07-solid}. I have a class that violates this principle,
\texttt{GameIO}. By the very nature of the class, it has two responsibilities:
player input and game output. This was intentional at the time as I had been
following a game design pattern I used in a previous project that used
a similar context (the context being a local multiplayer command-line Java game).
This decision did not support changeability from a design perspective, if we
consider that violating one of the SOLID principles reduces changeability. 

On the other hand, if we consider the changeability index (CI)
\cite{lec01b-changeability} for change cases that might involve changing player
input and game output, then we can consider that this approach provides
good changeability. The reason being that more classes are changed when
input/output are separated, and so that increases the CI if we assume that
other variables (total number of classes and impact of change case) are kept
constant.

Other aspects of my design also had issues, such as my decision to merge the
concepts of \textit{House} and \textit{Store} into a single \texttt{Pit} class.
This was highlighted 

\section{Understanding Changeability through Change Cases}
Although the changeability of a design can be evaluated through looking the
design principles being followed, ultimately, whether a design is changeable
can only be seen through the experience of attempts to change it. That is to
say, change cases. The reason why this is the case is that changeability is
concerned with how changeable a design is. Yet, if one never changes the
design, to talk about the changeability of it appears futile. For my design, 
there were two change cases that I successfully performed and both of them 
provided insights into changeability.

The first change case was concerned with modifying game output
\cite{a4-change-case} illustrated the changeability of my design. In that
reflection, I concluded that my design had good changeability due to the ease
with which I implemented that change case. I provided such arguments as ``[it]
resulted in minimal impact on the functionality of the implementation'', and it
can been seen in the change plan that the majority of the changed code occured
in a single method within a single class. If we then look at the changeability
index from that change case, this also supports the idea that there was good
changeability.
\begin{equation}
  CI = (3 / 7) * 0.1 = 0.04286
\end{equation}
Values for CI that fall below \texttt{1.0} are considered indicators of 'good'
changeability, and values that exceed \texttt{1.0} are considered indicators of
'bad' changeability. So the CI here tells us that for this change case there
was 'good' changeability. 

The second change case was concerned with adding a bot in place of the second
player, changing the game from multiplayer (with two human players) into
singleplayer (a human player against a computer player) \cite{a5-change-case}.
In implementing that change case, I ended up writing a large amount of new code
due to being unable to easily modify the pre-existing code. I then used this to
conclude that the changeability of my design could be improved, in spite of the
fact that very little of the pre-existing code was modified. This conclusion
\textit{did not} follow the definition of changeability provided in this
course. Again, if we consider the CI, we can see that the design has good
changeability.
\begin{equation}
  CI = (1 / 7) * 0.3 = 0.04286
\end{equation}
Yet, two change cases is not sufficient for claiming that a design has good
changeability. It is necessary to look at a variety of change cases in order to
truly assess changeability. Several additional change cases have been suggested
for Kalah, and from a rather extensive list I have narrowed ones that
I consider of great interest. Change cases that I do not consider interesting
and were thus excluded are ones that seem to be more concerned with changing
the fundamental rules of Kalah rather than exploring changes in program design
(i.e. increasing the number of players beyond 2, giving multiple rows to each
player, etc.). I also excluded change cases that were too similar to the ones
already discussed. While it is also worth noting that there are many potential
change case, the ones I will be considering are listed below.\newline
\begin{itemize}
  \item[\textbf{INF}] Tracking statistical information about the player, such
    as games won, average number of seeds in a winning game, average number of
    captures in a game, etc.
  \item[\textbf{S/L}] Save/load feature for individual games
  \item[\textbf{CR}] Different capture rules
  %\item[\textbf{DIR}] Direction (clockwise or counter-clockwise) that seeds are
  %  sown around the board
  \item[\textbf{NS}] Number of seeds per house at start of game
  \item[\textbf{NH}] Number of houses per player
\end{itemize}
The change cases \textbf{NS} and \textbf{NH} would be expected to be trivial
for most designs, and is in fact especially trivial in my design. They only require
a single edit each to one class, \texttt{GameConfig}. So we could confidently
say that for those change cases my design has good changeability, as there is
merely only 1 or 2 \textit{characters} (not lines) of code changed. However,
this should be expected from such simple change cases. If we then consider
more advance change cases, such as \textbf{INF} or \textbf{S/L}, the discussion
around the changeability of the design becomes more interesting. Both change
cases are concerned with preserving some data after the program exits. One
could imagine several implementations to achieve this, but I am going to
suggest ones for the sake of my argument. For \textbf{INF}, consider an
approach that writes to a file when certain events occur (such as
winning/losing a game or performing a capture). The original design would be
left largely untouched, save for some method injections which don't affect the
changeability as they are new code. For example, when determining the winner of
a game, it would be simple to add in a method to write that information to
a file before the return statement. Similar methods could be added in other
parts of the design, such as when a player receives a bonus turn or performs
a capture. For \textbf{S/L}, consider an approach that serialises objects into file storage, and is able to load them
from a file and deserialise them. 

By contrast, I could say that my design does not support \textbf{CR} very well due to the messy
implementation of captures. That is to say, my design does not make use of
a nicely written method called \texttt{executeCapture} or something similar,
but rather relies on a set of conditionals within a loop to determine when
a capture takes place. As such, this functionality is not isolated from other
possible moves, so a programmer implementing such a change case would need to
first wrap their head around my implementation. That being said, the definition
of changeability used here is only concerned with the execution of changes
\textit{once those changes have been identified}. A programmer could spend an
enormous amount of time figuring out those changes, but so long as the changes
were small, the design would still be considered changeable. With this in mind,
although I can look at the code and complain about the way it is written, it
would appear that there is in fact a good argument here for changeability.
After all, the implementation of a capture only exists within a single method
in a single class in my design.

\section{Concluding Statements}
After implemeting and modifying my design, it turns out that changeability is 
not as straightforward as I first thought. There are many approaches one can
take to ensuring good levels of changeability, and

\begin{thebibliography}{00}
  \bibitem{a3-design} M. Eden, ``Implementing Kalah with a Changeable
    Object-Oriented Design'', in Canvas, SOFTENG 701, Assignment 3 
    Submission
  \bibitem{a4-change-case} M. Eden, ``Implementing the Change Board Orientation
    change case'', in Canvas, SOFTENG 701, Assignment 4 Submission
  \bibitem{a5-change-case} M. Eden, ``Implementing the Best move or First robot
    change case'', in Canvas, SOFTENG 701, Assignment 5 Submission
  \bibitem{meyer} B. Meyer, ``Reusability: The Case for Object-Oriented
    Design'', in IEEE Software, vol. 4, no. 2, pp. 50-64, March 1987, 
    doi: 10.1109/MS.1987.230097
  \bibitem{lec01b-changeability} E. Tempero ``SOFTENG 701 - Lecture 01b - Changeability'',  in Canvas, SOFTENG 701, May 2020
  \bibitem{lec03-design} E. Tempero ``SOFTENG 701 - Lecture 03 - How to improve design'',  in Canvas, SOFTENG 701, May 2020
  \bibitem{lec04-measurement} E. Tempero ``SOFTENG 701 - Lecture 04 - Learning to Measure'',  in Canvas, SOFTENG 701, May 2020
  \bibitem{lec07-solid} E. Tempero ``SOFTENG 701 - Lecture 07 - SOLID Design Prinicples'',  in Canvas, SOFTENG 701, May 2020
  \bibitem{lec09-counting} E. Tempero ``SOFTENG 701 - Lecture 09 - Counting Objects to evaluate object-oriented design'',  in Canvas, SOFTENG 701, June 2020
\end{thebibliography}
\end{document}
