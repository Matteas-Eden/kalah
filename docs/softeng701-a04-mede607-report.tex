\documentclass[10pt, a4paper, conference]{IEEEtran}
%\IEEEoverridecommandlockouts
% The preceding line is only needed to identify funding in the first footnote. If that is unneeded, please comment it out.
\usepackage{cite}
\usepackage{amsmath,amssymb,amsfonts}
\usepackage{graphicx}
\usepackage{textcomp}
\usepackage{xcolor}
\def\BibTeX{{\rm B\kern-.05em{\sc i\kern-.025em b}\kern-.08em
    T\kern-.1667em\lower.7ex\hbox{E}\kern-.125emX}}
\begin{document}

\title{Implementing the \textit{Change Board Orientation} change case}

\author{\IEEEauthorblockN{Matthew Eden}
\IEEEauthorblockA{\textit{Department of Electrical, Computer and Software
Engineering} \\
\textit{University of Auckland}\\
Auckland, New Zealand \\
mede607@aucklanduni.ac.nz}
}

\maketitle

\begin{abstract}
  To evaluate the changeability of an implementation of Kalah (a.k.a Mancala), a change case was
  proposed. This involved changing the \textit{horizontal} orientation of the board
  to a \textit{vertical} orientation. A change plan was created in anticipation
  of implementing the changes in code. During the execution of this plan, there
  were some issues encountered, but not a significant amount. The time taken to implement
  the change case was roughly 16 minutes. The overall impact of the change case was
  low and it was straightforward to preserve the functionality for the horizontal
  orientation if this change need be reverted in future. As such, there is a good
  amount of changeability in this design.
\end{abstract}

\begin{IEEEkeywords}
kalah, mancala, changeability, java
\end{IEEEkeywords}

\section{Planning the Change Case}
Before any code was actually written, a plan was devised describing the minimal
amount of changes that would be required to achieve the desired change case.
This plan is as follows.
\begin{enumerate}
  \item Find the \texttt{GameIO} class
  \item Locate the \texttt{printBoard} method
  \item Add an additional argument: \texttt{boolean isVertical}
  \item With the exception of the first non-blank line concerning
    \texttt{List<Players>}, wrap the method body in an \texttt{if-else} statement.
  \item Make this \texttt{if} condition controlled by the new method argument, such that
    \texttt{if (isVertical) \{...\} else \{/* ORIGINAL METHOD BODY */\}}
  \item Within the \texttt{if} branch, add 4 \texttt{io.println} statements for
    printing out the outline of the board and the player stores, ensuring that
    they follow the specified format.
  \item In the middle of those statements (i.e. between the printing of player
    stores), add a call to a new method
    \texttt{printHousesVertically}, passing in \texttt{GameConfig.NUM\_HOUSES}
    as the argument.
  \item Implement this method to print out the houses of both players such that
    it follows the specifed format.
  \item Update invocations of \texttt{printBoard} to pass in \texttt{true} as a third argument.
\end{enumerate}

\section{Implementing the Change Case}
The total amount of time taken to implement the change case was timed as being 
15 minutes and 58 seconds. During the execution of the aforementioned plan, there were some
issues encountered. Overall, the plan was a success and it was not difficult to
implement the change case. A breakdown of the steps carried out follows. 
\begin{enumerate}
  \item No issues in finding the \texttt{GameIO} class.
  \item No issues in locating the \texttt{printBoard} method.
  \item No issues in adding an additional argument to \texttt{printBoard}.
  \item No issues in creating an \texttt{if-else} statement to wrap the method
    body.
  \item No issues in implementing the condition for the \texttt{if-else}
    statement.
  \item No issues in adding the required \texttt{io.println} statements.
    Although the argument to be provided to those statements was not specified
    (i.e. what exactly should be printed), it was not difficult to figure it
    out based on the format specified by the test cases and the pre-existing
    implementation of the \texttt{printBoard} method. 
  \item No issues in adding a call to \texttt{printHousesVertically} and
    providing \texttt{GameConfig.NUM\_HOUSES} as the argument.
  \item There was some issue in implementing \texttt{printHousesVertically}.
    The number of method arguments specified in the plan was not sufficient, as
    the final version of the method created required \textit{three} arguments, as opposed to
    the \textit{one} that was originally planned. The two additional arguments
    were \texttt{IO io} and \texttt{List<Pit> pits}. This then meant that the
    invocation of \texttt{printHousesVertically} in \texttt{printBoard}
    required changing and required additional values to be passed in. There was
    already a pre-existing \texttt{IO io} argument in \texttt{printBoard}, so there was no
    issue in passing that as an argument to \texttt{printHousesVertically}.
    However, there was no pre-existing \texttt{List<Pit> pits} or equivalent
    argument. Within the \texttt{printBoard} method there was a pre-existing
    \texttt{Board board} argument, with that object containing
    a \texttt{List<Pit> pits} field that would be sufficient. However, there did not
    exist a getter method to access that field, so this had to be
    implemented. The plan did also not go into detail about \textit{how} to
    implement the \texttt{printHousesVertically} method, however the
    pre-existing method \texttt{formatHousesAsString} was sufficient as a guide for
    figuring out the implementation.
  \item No issues in updating existing invocations of 
    \texttt{printBoard} to pass in \texttt{true} as a third argument.
\end{enumerate}

\section{Impact of the Change Case}
On a scale from \textit{0.0} to \textit{1.0}, the impact of this change case is
\textit{0.1}. For all intents and purposes, the game of \textit{Kalah} can be
played in the same manner as before the change case despite the visual change.
In saying that, these visual changes mean the impact is strictly non-zero.

\section{Changeability Assessment}
My original assessment of the changeability of this design claimed that the
design supported changeability in part because of its separation of IO and game
logic \cite{a3-design}. The successful implemention of the \textit{Change Board
Orientation} change case further supports those claims. Although there were
some issues faced when following the change plan, these issues could be argued to
arise from poor planning rather than a lack of changeability in the design. The
only issue directly related to \textit{changeability} that arose was the lack
of a \texttt{getPits} method in the \texttt{Board} class. In spite of this,
the implementation of the change case did not take a significant
amount of time and resulted in minimal impact on the functionality of the
implementation. As such, the original changeability assessment of the design
was correct.

\begin{thebibliography}{00}
  \bibitem{a3-design} M. Eden ``SOFTENG 701 - Assignment 3 - Submission``,
    \textit{Implementing Kalah with a Changeable Object-Oriented Design}
\end{thebibliography}
\end{document}
