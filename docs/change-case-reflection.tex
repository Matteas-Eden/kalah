\documentclass[10pt, a4paper, conference]{IEEEtran}
%\IEEEoverridecommandlockouts
% The preceding line is only needed to identify funding in the first footnote. If that is unneeded, please comment it out.
\usepackage{cite}
\usepackage{amsmath,amssymb,amsfonts}
\usepackage{graphicx}
\usepackage{textcomp}
\usepackage{xcolor}
\def\BibTeX{{\rm B\kern-.05em{\sc i\kern-.025em b}\kern-.08em
    T\kern-.1667em\lower.7ex\hbox{E}\kern-.125emX}}
\begin{document}

\title{Implementing the \textit{Change Board Orientation} change case}
}

\author{\IEEEauthorblockN{Matthew Eden}
\IEEEauthorblockA{\textit{Department of Electrical, Computer and Software
Engineering} \\
\textit{University of Auckland}\\
Auckland, New Zealand \\
mede607@aucklanduni.ac.nz}
}

\maketitle

\begin{abstract}
  To evaluate the changeability of an implementation of Kalah (a.k.a Mancala), a change case was
  proposed. This involved changing the \textit{horizontal} orientation of the board
  to a \textit{vertical} orientation. A change plan was created in anticipation
  of implementing the changes in code. During the execution of this plan, there
  was some deviation, but not a significant amount. The time taken to implement
  the change case was roughly 15 minutes. The overall impact of the change was
  low and it was straightforward to preserve the functionality for the horizontal
  orientation if this change needed to be reverted. As such, there is a good
  amount of changeability in this design.
\end{abstract}

\begin{IEEEkeywords}
kalah, mancala, changeability, java
\end{IEEEkeywords}

\section{Change Plan}
\section{Deviations from the Plan}
\section{Impact of the Change Case}
\section{Changeability Assessment}

\end{document}
